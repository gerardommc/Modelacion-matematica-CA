% Options for packages loaded elsewhere
\PassOptionsToPackage{unicode}{hyperref}
\PassOptionsToPackage{hyphens}{url}
%
\documentclass[
]{book}
\usepackage{amsmath,amssymb}
\usepackage{lmodern}
\usepackage{ifxetex,ifluatex}
\ifnum 0\ifxetex 1\fi\ifluatex 1\fi=0 % if pdftex
  \usepackage[T1]{fontenc}
  \usepackage[utf8]{inputenc}
  \usepackage{textcomp} % provide euro and other symbols
\else % if luatex or xetex
  \usepackage{unicode-math}
  \defaultfontfeatures{Scale=MatchLowercase}
  \defaultfontfeatures[\rmfamily]{Ligatures=TeX,Scale=1}
\fi
% Use upquote if available, for straight quotes in verbatim environments
\IfFileExists{upquote.sty}{\usepackage{upquote}}{}
\IfFileExists{microtype.sty}{% use microtype if available
  \usepackage[]{microtype}
  \UseMicrotypeSet[protrusion]{basicmath} % disable protrusion for tt fonts
}{}
\makeatletter
\@ifundefined{KOMAClassName}{% if non-KOMA class
  \IfFileExists{parskip.sty}{%
    \usepackage{parskip}
  }{% else
    \setlength{\parindent}{0pt}
    \setlength{\parskip}{6pt plus 2pt minus 1pt}}
}{% if KOMA class
  \KOMAoptions{parskip=half}}
\makeatother
\usepackage{xcolor}
\IfFileExists{xurl.sty}{\usepackage{xurl}}{} % add URL line breaks if available
\IfFileExists{bookmark.sty}{\usepackage{bookmark}}{\usepackage{hyperref}}
\hypersetup{
  pdftitle={Modelación matemática},
  pdfauthor={Gerardo Martín},
  hidelinks,
  pdfcreator={LaTeX via pandoc}}
\urlstyle{same} % disable monospaced font for URLs
\usepackage{longtable,booktabs,array}
\usepackage{calc} % for calculating minipage widths
% Correct order of tables after \paragraph or \subparagraph
\usepackage{etoolbox}
\makeatletter
\patchcmd\longtable{\par}{\if@noskipsec\mbox{}\fi\par}{}{}
\makeatother
% Allow footnotes in longtable head/foot
\IfFileExists{footnotehyper.sty}{\usepackage{footnotehyper}}{\usepackage{footnote}}
\makesavenoteenv{longtable}
\usepackage{graphicx}
\makeatletter
\def\maxwidth{\ifdim\Gin@nat@width>\linewidth\linewidth\else\Gin@nat@width\fi}
\def\maxheight{\ifdim\Gin@nat@height>\textheight\textheight\else\Gin@nat@height\fi}
\makeatother
% Scale images if necessary, so that they will not overflow the page
% margins by default, and it is still possible to overwrite the defaults
% using explicit options in \includegraphics[width, height, ...]{}
\setkeys{Gin}{width=\maxwidth,height=\maxheight,keepaspectratio}
% Set default figure placement to htbp
\makeatletter
\def\fps@figure{htbp}
\makeatother
\setlength{\emergencystretch}{3em} % prevent overfull lines
\providecommand{\tightlist}{%
  \setlength{\itemsep}{0pt}\setlength{\parskip}{0pt}}
\setcounter{secnumdepth}{5}
\usepackage{booktabs}
\ifluatex
  \usepackage{selnolig}  % disable illegal ligatures
\fi
\usepackage[]{natbib}
\bibliographystyle{apalike}

\title{Modelación matemática}
\author{Gerardo Martín}
\date{2021-07-29}

\begin{document}
\maketitle

{
\setcounter{tocdepth}{1}
\tableofcontents
}
\hypertarget{sobre-este-curso}{%
\chapter{Sobre este curso}\label{sobre-este-curso}}

En el curso \textbf{Modelación matemática} aprenderemos a utilizar algunas herramientas matemáticas para analizar y entender los procesos de interés en ciencias ambientales. Los contenidos del índice se apegan al \href{Programa-curso.pdf}{programa completo del curso}, el cual se impartirá en los \href{Horario.pdf}{horarios normales establecidos}. Para conocer cuándo, cómo y qué temas se se impartirán puedes consultar la \href{Estrategia-docente.pdf}{estrategia docente}.

\hypertarget{criterios-de-evaluaciuxf3n}{%
\chapter{Criterios de evaluación}\label{criterios-de-evaluaciuxf3n}}

Las constribuciones a cada calificación parcial serán por igual (25\% cada uno):

\begin{itemize}
\tightlist
\item
  Asistencia
\item
  Trabajos de clase cumplidos
\item
  Participación
\item
  Examen
\end{itemize}

\hypertarget{unidad-i-modelos-determinuxedsticos}{%
\chapter{Unidad I: Modelos determinísticos}\label{unidad-i-modelos-determinuxedsticos}}

\hypertarget{funciones-buxe1sicas-y-su-representaciuxf3n-en-el-plano-cartesiano}{%
\section{Funciones básicas y su representación en el plano cartesiano}\label{funciones-buxe1sicas-y-su-representaciuxf3n-en-el-plano-cartesiano}}

\hypertarget{la-luxednea-recta}{%
\subsection{La línea recta}\label{la-luxednea-recta}}

La línea recta es\ldots{}

\hypertarget{funciones-complementarias-y-su-representaciuxf3n-en-dos-y-tres-dimensiones}{%
\section{Funciones complementarias y su representación en dos y tres dimensiones}\label{funciones-complementarias-y-su-representaciuxf3n-en-dos-y-tres-dimensiones}}

\hypertarget{la-luxednea-recta-como-modelo-universal}{%
\section{La línea recta como modelo ``universal''}\label{la-luxednea-recta-como-modelo-universal}}

\hypertarget{modelaciuxf3n-de-sistemas-sociales-y-ambientales}{%
\section{Modelación de sistemas sociales y ambientales}\label{modelaciuxf3n-de-sistemas-sociales-y-ambientales}}

\hypertarget{methods}{%
\chapter{Methods}\label{methods}}

We describe our methods in this chapter.

\hypertarget{applications}{%
\chapter{Applications}\label{applications}}

Some \emph{significant} applications are demonstrated in this chapter.

\hypertarget{example-one}{%
\section{Example one}\label{example-one}}

\hypertarget{example-two}{%
\section{Example two}\label{example-two}}

  \bibliography{book.bib,packages.bib}

\end{document}
